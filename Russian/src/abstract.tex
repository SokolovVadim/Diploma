\newpage
\section*{Аннотация}

\textbf{Цели и задачи работы.}

Данная работа посвящена исследованию структур данных,
использующих сжатые индексы (succinct index) для хранения текстовой информации.

Целью данной работы является проверка эффективности различных методов сжатия данных.
Исследуется применимость сжатых индексов на практике при работе с данными определенного типа.
Производится сравнение как с традиционными решениями (suffix array), так и с более современными (radix tree),
использующими подход для индексирования, отличный от исследуемых структур данных.


\textbf{Полученные результаты.}


Удалось получить сравнительные характеристики работы исследуемых структур данных.
Были измерены и проанализированы:
\begin{enumerate}
    \item объем потребляемой памяти;
    \item время, требуемое для поиска подстроки;
\end{enumerate}

На языке Go реализована сжатая структура данных succinct suffix array. Приведены результаты потребления памяти
для хранения сжатого индекса и скорости поиска подстроки в тексте.


