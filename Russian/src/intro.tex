
% inverted index
% bioinformatics is important
% pattern matching is used in it
% so it is crusial to work with substrings

Работа с текстовыми данными находит применение в широком спектре задач современной компьютерной индустрии.
При работе с текстовыми данными очень часто используется inverted index \cite{zobel2006inverted}. Эта структура данных
позволяет индексировать слова в документах. Но существует ряд проблем, для которых индексирование слов
не является решением.
Например, в ряде задач биоинформатики \cite{tsuruoka2008facta} присутствует поиск паттерна
в ДНК-коде или коде белковой структуры,
которые не представляют собой слова и соответственно inverted index для них не применим.
Аналогично в некоторых восточных языках (китайский, арабский) текст не делится на отдельные слова.
Дополнительно, inverted index ограничен при поиске строки, похожей на искомую, но не совпадающую с ней в точности.
Например, при поиске информации в поисковых системах, где в качестве запроса используется текст, содержащий
орфографические ошибки или другие неточности, inverted index не сможет дать пользователю искомый ответ.
В этих случаях используются индексы, построенные на отсортированных суффиксах исходного текста.
Примером такого индекса является суффиксный массив (suffix array) \cite{manber1993suffix}.
В нем реализована возможность поиска подстроки,
что позволяет решать задачи, недоступные для inverted index. В свою очередь, suffix array может занимать
место, превосходящее исходный текст до 50 раз.
Поэтому может оказаться выгоднее выполнять простое сканирование по тексту в RAM,
чем иметь индекс, который в RAM не помещается, т.к. операции по чтению / записи информации во внешней памяти
являются значительно более затратными.

Рост количества информации в Интернете приводит к дополнительным издержкам при хранении и поиске данных.
В связи с этим существует необходимость исследования различных способов уменьшения потребляемой памяти без
существенных затрат на поиск данных.

Одним из возможных решений такого рода задач является применение сжатых структур данных
(succinct data structures) \cite{jacobson1988succinct}.
В зависимости от степени сжатия информации структуры данных различаются на имплицитные, сжатые и компактные.
Сжатые структуры используют близкое к теоретически минимальному количеству информации для хранения данных.
Кроме того, в отличие от архивов и других сжатых представлений, остается возможность
эффективно выполнять операции поиска.
Предположим, что для хранения некоторого количества данных требуется Z бит.
Сжатые структуры данных занимают \(Z + o(Z)\) бит. Например структура данных, занимающая \(Z + \ln(Z)\) бит памяти,
является сжатой.

Данные не всегда сжимаемы. Кроме того, не любые данные целесообразно сжимать с точки зрения эффективности
их использования в несжатом виде. В этой работе предлагается рассмотреть сжатие индекса суффиксного массива,
построенного для различных тектовых данных. При этом сам текст остается в несжатом виде.

Для того чтобы представить данные в сжатом виде, необходимо подготовить их
в специальном промежуточном формате. В этой работе используется алгоритм Элиас-Фано \cite{pibiri2014dynamic},
позволяющий сжимать возрастающие последовательности неотрицательных целых чисел.

В ходе разработки структуры данных сжатого суффиксного массива была задействована оптимизация
по памяти, заключающаяся в добавлении индексации по битмапу, позволяющая использовать меньше памяти,
чем аналоги, использующие вспомогательные структуры данных.

Целью исследования является изучение потребления памяти для сжатого представления суффиксного массива
на текстах различного содержания и сравнение с существующими подходами.
Реализованы функции поиска подстроки, и произведен анализ их эффективности.

