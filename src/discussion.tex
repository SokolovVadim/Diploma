% SA
% Radix
% CSA
% Comparison

Рассмотренная структура данных, сжатый суффиксный массив (CSA), показала свою эффективность
с точки зрения памяти, занимаемой для хранения индекса. Относительно традиционного
suffix array, как было показано на графиках сравнительных характеристик,
при достаточно большом размере исходного текста CSA позволяет использовать меньше памяти, чем
suffix array, сохраняя при этом возможность осуществлять поиск подстроки.
Как было замечено при обсуждении зависимости сжатия от длины текста,
при малых значени размера входного текста suffix array является более эффективным с точки
зрения потребляемой памяти, что вызвано дополнительными затратами на обслуживание
вспомогательных структур данных, использующихся для построения сжатого индекса.
В то же время, как алгоритм построения индекса, так и поиска подстроки в CSA работают
медленее, чем в исходном. Тем не менее, существуют методы улучшения скорости
работы этих алгоритмов, позволяющие CSA достигнуть результатов, не уступающих suffix array.

Результаты, полученные при сравнении radix tree с остальными исследуемыми структурами данных,
подтверждают эффективность поиска подстроки в radix tree за константное время. Однако
для индексации по тексту необходимо заполнить дерево подстроками исходного текста,
занимающими значительно большие объемы памяти по сравнению с suffix array и CSA.
Аналогично, скорость заполнения дерева при построении не позволяет считать radix tree
эффективным при работе с подстроками исходного текста.
