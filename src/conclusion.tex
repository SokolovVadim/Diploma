% what have been done
% what have been proved
% what are the further research topics to be covered

На языке Go была разработана структура данных сжатый суффиксный массив (CSA), имеющая
функционал традиционного suffix array, позволяеющего решать задачи поиска по подстрокам.
Реализован алгоритм Elias-Fano, позволяющий представлять индекс в сжатом виде.
Построены сравнительные характеристики работы по построению и поиску подстроки
для CSA и suffix array для текстов различного содержания.
Изучена зависимость потребляемой памяти структурами данных от размеров исходного текста.
Произведено сравнение radix tree с этими структурами по времени построения и поиска подстроки,
а также по затраченной памяти.

С точки зрения потребляемой памяти CSA показал свою эффективность по сранению с suffix array и radix tree,
но разработанный алгоритм построения индекса и поиска подстроки является менее быстрым,
чем аналогичные алгоритмы в традиционном suffix array. Показана эффективность поиска подстроки
при помощи radix tree, подчеркнута сложность его использования для текстовых данных из-за больших
затрат памяти при построении дерева.

Дальнейшие исследования в этой области могут опираться на разработку сжатых представлений существующих
классических структур данных. Применение методов сжатия,
таких как рассмотренный в этой работе алгоритм Elias-Fano, может позволить производить сжатое представление
данных различного типа, предварительно преобразованных к нужному формату.
Сжатые структуры данных могут быть задействованы в работе алгоритмов,
использующих хранение индекса во внешней памяти, что может стать предметом дополнительных исследований.
