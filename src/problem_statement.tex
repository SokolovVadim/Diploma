
Прежде всего, сжатое представление требует ряда манипуляций над исходным суффиксным массивом.
В связи с этим появляется необходимость использования промежуточного $\psi$-массива.
Он представляет собой реорганизованный суффиксный массив,
в котором присутствует несколько наборов возрастающих последовательностей индексов.

Возрастающая последовательность целых неотрицательных чисел сжимаема, что было показано в работе \cite{pibiri2014dynamic}
Для сжатия такой последовательности применяется алгоритм Элиаса--Фано, позволяющий записать
$\psi$-массив в виде битового вектора.

Особенностью представления Элиаса--Фано является возможность восстановления значения $\psi$-массива
по индексу. Для битовых векторов определены операции \emph{rank} и \emph{select},
работающие за константное время farina2009rank.

Непосредственными задачами являются:

\begin{enumerate}
    \item Построение суффиксного массива для данного текста (набора символов).
    \item Построение $\psi$-массива по полученному суффиксному массиву.
    \item Реализация алгоритма сжатия Элиаса-Фано.
    \item Написание функции получения индекса в суффиксном массиве по $\psi$-представлению.
    \item Реализация алгоритма распаковки индекса в $\psi$-массиве.
    \item Написание функции поиска подстроки в исходном тексте.
    \item Написание бенчмарков для сравнения сжатого суффиксного массива
    с традиционным суффиксным массивом и radix tree.
\end{enumerate}


