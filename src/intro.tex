
Работа с текстовыми данными находит применение в широком спектре задач современной компьютерной индустрии.
Существует ряд проблем, связанных с поиском информации в поисковых сервисах.
Рост количества информации в Интернете приводит к дополнительным издержкам при хранении и поиске данных.
В связи с этим существует необходимость исследования различных способов уменьшения потребляемой памяти без
существенных затрат на поиск данных.


Одним из возможных решений такого рода задач является применение сжатых структур данных (succinct data structures).
В зависимости от степени сжатия информации структуры данных различаются на имплицитные, сжатые и компактные.
Сжатые структуры используют близкое к теоретически минимальному количеству информации для хранения данных.
Кроме того, в отличие от архивов и других сжатых представлений, остается возможность
эффективно выполнять операции поиска.
Предположим, что для хранения некоторого количества данных требуется Z бит.
Сжатые структуры данных занимают \(Z + o(Z)\) бит. Например структура данных, занимающая \(Z + \ln(Z)\) бит памяти,
является сжатой.


Данные не всегда сжимаемы. Кроме того, не любые данные целесообразно сжимать с точки зрения эффективности
их использования в несжатом виде. В этой работе предлагается рассмотреть сжатие индекса суффиксного массива,
построенного для различных тектовых данных. При этом сам текст остается в несжатом виде.


Для того чтобы представить данные в сжатом виде, необходимо подготовить их
в специальном промежуточном формате. В этой работе используется алгоритм Элиас-Фано,
позволяющий сжимать возрастающие последовательности неотрицательных целых чисел.
Исследование направлено на изучение потребления памяти для сжатого представления суффиксного массива.
Реализованы функции поиска подстроки, и произведен анализ их эффективности.

