% overview +
% complexity +
% code snippet +
% measurements +
% table +

In order to analyze the data structure, suffix array from a standart Go library
was taken into consideration \cite{golang2016sa}. It takes as an input
a text and constructs a sorted index from it with that it is possible to perform
substring search operations. Index construction takes $O(n)$ operations, where $n$ is
the initial text size. Substring search takes $O(\log n \cdot |s|)$, where $|s|$ is the substring length.
Let us consider benchmarks more carefully. The initial text is loaded from the file,
then the substring is chosen that is limited by positions $[leftPos:rigtPos]$.

\begin{lstlisting}[caption=Suffix array example]
func BenchmarkLookup(b *testing.B, testStr []byte) {
    sa := suffixarray.New(testStr)
    b.ResetTimer()
    for i := 0; i < b.N; i++ {
        offset := sa.Lookup(testStr[leftPos: rightPos], -1)
        if len(offset) < 1 || offset[0] != leftPos {
            b.Fatalf("mis-match: %v", offset)
        }
    }
}
\end{lstlisting}

Listing 1 shows a simplified code of a benchmark for a substring search in a suffix array
(less important for general understanding detailes are skipped). The code is tested for 5 texts with
different contents. Substring size for data search is the same for each measurement.

\begin{table}[ht!]
    \centering
    \begin{tabular}{|c|c|c|}
        \hline
        Text size, KB & Memory, KB & Time, s\\
        \hline
        9766 & 58741 & 1.832\\
        \hline
        977 & 6020 & 0.337\\
        \hline
        879 & 5457 & 0.333\\
        \hline
        782 & 4838 & 0.323\\
        \hline
        684 & 4252 & 0.314\\
        \hline
        586 & 3669 & 0.305\\
        \hline
        489 & 3589 & 0.297\\
        \hline
        391 & 2925 & 0.285\\
        \hline
        293 & 2244 & 0.275\\
        \hline
        196 & 1563 & 0.274\\
        \hline
        98 & 881 & 0.260\\
        \hline
        49 & 547 & 0.258\\
        \hline
        10 & 246 & 0.252\\
        \hline
        5 & 212 & 0.251\\
        \hline
        2 & 194 & 0.256\\
        \hline
    \end{tabular}
    \caption{Suffix array construction}
    \label{table:2}
\end{table}

Because suffix array size does not depend on the alphabet size
the results for different texts vary within the margin of error.
That is why it is enough to show for clarity an example for the same text type.
Table \ref{table:2} shows the results of measurements of the index construction for the Amazon Text Corpora.

% size, kb: 977, 879, 782, 684, 586, 489, 391, 293, 196, 98
% time, mks: 12251, 11857, 12331, 12969, 12777, 12680, 11795, 11555, 11752, 11657
% mem, kb:  6101, 5501, 4915, 4333, 3747, 3707, 3001, 2322, 1642, 962

\begin{table}[ht!]
    \centering
    \begin{tabular}{|c|c|c|}
        \hline
        Text size, KB & Memory, KB & Time, ms\\
        \hline
        977 & 6020 & 12.251\\
        \hline
        879 & 5457 & 11.857\\
        \hline
        782 & 4838 & 12.331\\
        \hline
        684 & 4252 & 12.969\\
        \hline
        586 & 3669 & 12.777\\
        \hline
        489 & 3589 & 12.680\\
        \hline
        391 & 2925 & 11.795\\
        \hline
        293 & 2244 & 11.555\\
        \hline
        196 & 1563 & 11.752\\
        \hline
        98 & 881 & 11.657\\
        \hline
    \end{tabular}
    \caption{Substring search in suffix array}
    \label{table:3}
\end{table}
