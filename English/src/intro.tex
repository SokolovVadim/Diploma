
% inverted index
% bioinformatics is important
% pattern matching is used in it
% so it is crusial to work with substrings

Text data processing is used in a wide range of tasks in the modern computer industry.
Dealing with text data inverted index \cite{zobel2006inverted} is commonly used. This data structure
allows to index words in documents.
But there is a number of problems for which word indexing is not a good solution.
For example, in some bioinformatics problems \cite{tsuruoka2008facta} pattern search takes place
in DNA-code or protein sequence structure, which are not words and therefore inverted index
can't be applied in this case.
Similarly in some Eastern languages (Chinese, Arabic) text can not be devided in separated words.
In addition, inverted index is limited when it comes to search for a string similar to required,
but not totally identical to it. As an example, during the information search via search engines,
where text with spelling mistakes is used as a request, inverted index can not give the right reply.
In these cases indexes are built on unsorted suffixes of the initial text.
The example of this index is a suffix array \cite{manber1993suffix}.
Substring search is implemented in a suffix array, which allows us to resolve the problems
that cannot be solved by inverted index.
In the contrary, suffix array can take 50 times more space than the initial text.
Therefore it might be more cost-effective to execute simple text scaning in RAM 
instead of having an index that does not fit in RAM,
because I/O operations in outer memory are significantly more cost-intensive.

The increase in information search on the Internet leads to additional expenses in data storage and search.
In this regard there is a necessity in research of different methods of reduction in consuming memory without 
significant costs on data search.

One of possible solutions for this type of problems is application
of succinct data structures \cite{jacobson1988succinct}.
Depending on the level of information compression data structures can be divided by implicit, succinct and compact.
Succinct data structures use the information volume that is close to theoretically minimal.
In addition to that, in contrast to archives and other compressed solutions 
it is possible to provide effective search operations.
Let assume that it takes Z bit for storing a certain amount of data.
Succinct data structures take \(Z + o(Z)\) bit. 
For example, data structure that takes \(Z + \ln(Z)\) bit of memory is succinct.

Data is not always compressible. Apart from that not all the data is reasonable to compress
in terms of efficiency of it in uncompressed state.
This paper proposes to cosider compression of index of suffix array,
built for different textual data. At the same time, text stays uncompressed.

In order to present the data in a compressed state it is necessary to prepare
data in a special intermediate format.
In this paper Elias-Fano encoding algorithm \cite{pibiri2014dynamic} is used.
This algorithm allows us to compress an ascending sequence of non-negative integers.

While developing data structure of succinct suffix array memory optimization was used consisting in adding indexaton
over a bitmap that makes possible to use less memory than similar approaches that use additional data structures.

The goal of this research is to study memory consumption for succinct suffix array over texts of various content
and comparison with existing approaches.
Substring search functions have been developed and the analysis of their efficiency has been carried out.
