First of all, compressed representation requires several manipulations over the inital suffix array.
Because of that, there is a necessity to use intermediate $\psi$-array.
This array implies a reorganized suffix array that consists of several sets of ascending sequences of indexes.

Ascending sequence of non-negative integers is compressible, which was shown in this paper \cite{pibiri2014dynamic}.
In order to compress this sequence it is necessary to use Elias--Fano algorithm, which makes it possible to
write $\psi$-array in a bit vector representation.

One of the characteristics of Elias--Fano representation is the opportunity to recover the values of $\psi$-array
by index. Operations \emph{rank} and \emph{select} are defined for bit vectors,
that work for constant time \cite{farina2009rank}.

Objectives of the work:

\begin{enumerate}
    \item Building of suffix array for given text (symbolic sequence).
    \item Building of $\psi$-array over constructed suffix array.
    \item Implementation of Elias--Fano compression algorithm.
    \item Implementation of a function for getting an index in suffix array by $\psi$-representation.
    \item Implementation of the index unpacking algorithm in $\psi$-array.
    \item Implementation of a function for a substring search in the initial text.
    \item Implementation of benchmarks for comparison of succinct suffix array with
    traditional suffix array and radix tree.
\end{enumerate}
