% what have been done
% what have been proved
% what are the further research topics to be covered

Using Go language data structure compressed suffix array (CSA) was implemented.
It has a functionality of a traditional suffix array allowing one to solve
problems of substring lookup. Elias-Fano algorithm was implemented allowing one
to present an index in a succinct representation. Comparative characteristics of
the work for construction and substring lookup for CSA and suffix array were combined
for texts with different contexts.
The dependency between memory consumed by data structures and the size of the initial text
was studied. A comparison between radix tree and these data structures was performed
by construction time and substring lookup time as well as by memory consumed.

In terms of memory used CSA showed it's efficiency in comparison to suffix array and radix tree,
but the algorithm of index construction and the substring lookup algorithm are less fast
than similar algorithms in a traditional suffix array. Application of the CSA at real data
was considered at applied problems such as substring lookup at a protein sequence and
pattern matching at DNA code used in bioinformatics.
The efficiency of substring search using radix tree was demonstrated. The complexity
of its application was shown for textual data because of large data consumption at tree construction step.

Futher studies in this research field can rely on a development of succinct representation of
existing classical data structures.

На языке Go была разработана структура данных сжатый суффиксный массив (CSA), имеющая
функционал традиционного suffix array, позволяющего решать задачи поиска по подстрокам.
Реализован алгоритм Elias-Fano, позволяющий представлять индекс в сжатом виде.
Построены сравнительные характеристики работы по построению и поиску подстроки
для CSA и suffix array для текстов различного содержания.
Изучена зависимость потребляемой памяти структурами данных от размеров исходного текста.
Произведено сравнение radix tree с этими структурами по времени построения и поиска подстроки,
а также по затраченной памяти.

С точки зрения потребляемой памяти CSA показал свою эффективность по сравнению с suffix array и radix tree,
но разработанный алгоритм построения индекса и поиска подстроки является менее быстрым,
чем аналогичные алгоритмы в традиционном suffix array. Рассмотрена применимость CSA на реальных
прикладных задачах, в том числе поиск подстроки в белковой последовательности
и pattern matching в ДНК-коде, используемые в биоинформатике.
Показана эффективность поиска подстроки при помощи radix tree,
подчеркнута сложность его использования для текстовых данных из-за больших
затрат памяти при построении дерева.

Дальнейшие исследования в этой области могут опираться на разработку сжатых представлений существующих
классических структур данных. Применение методов сжатия,
таких как рассмотренный в этой работе алгоритм Elias-Fano, может позволить производить сжатое представление
данных различного типа, предварительно преобразованных к нужному формату.
На основании экспериментов выявлены достоинства и недостатки исследуемых алгоритмов и структур данных
для работы над реальными прикладными задачами.
Сжатые структуры данных могут быть задействованы в работе алгоритмов,
использующих хранение индекса во внешней памяти. Эта тема является малоизученной, поэтому
сжатие индекса представляет особый интерес в области индексирования по подстрокам во внешней памяти.

