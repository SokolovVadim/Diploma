Radix tree implementation in Go language is taken from the following source: \cite{golang2016sa}
The main reason of this decision is caused by the simplisity of insertion of the benchmark
in the existing testing system in Go language.
From the algorithmic point of view this implementation confirms the theoretical description of
a substring search complexity that is $O(1)$.

\newpage
\begin{lstlisting}[caption=Radix tree example]
func BenchmarkConstruct(b *testing.B, testStr string) {
	var substringArray []string = createSubstrings(testStr)
	b.ResetTimer()
	r := radix.New()
	for i := 0; i < b.n; i++ {
		fillRadixTree(b.N, r, substringArray)
	}
}
\end{lstlisting}

Listing 2 briefly shows the radix tree construction function.
In the beginning a set of substrings is generated from the text that is read from file.
In $fillRadixTree$ function substrings are added to the tree. Comparison of
radix tree construction for Amazon Text Corpora \cite{amazon2013text} are shown in a table \ref{table:4}.

% size, kB: 196, 148, 98, 79, 49, 10, 5, 2
% time, s:  276.419, 128.494, 57.319, 36.786, 14.632, 0.845, 0.414, 0.284
% mem, KB:  20248160, 11501051, 5192071, 3353864, 1324283, 52803, 13528, 2314

\begin{table}[h!]
    \centering
    \begin{tabular}{|c|c|c|}
        \hline
        Text size, Kb & Memory, Mb & Time, s\\
        \hline
        196 & 20248 & 276.419\\
        \hline
        148 & 11501 & 128.494\\
        \hline
        98 & 5192 & 57.319\\
        \hline
        79 & 3353 & 36.786\\
        \hline
        49 & 1324 & 14.632\\
        \hline
        10 & 52 & 0.845\\
        \hline
        5 & 13 & 0.414\\
        \hline
        2 & 2 & 0.284\\
        \hline
    \end{tabular}
    \caption{Radix tree construction}
    \label{table:4}
\end{table}

In the listing 3 below the example of a tree substring search function is given.
This data structure shows quite good results of a substring search in a constant time.
It can be seen at the table \ref{table:5}.

\newpage
\begin{lstlisting}[caption=Radix tree lookup]
func getSubstring(r *radix.Tree,
                  subString string, testStr string) {
	var out interface{}
	var pos interface{}
	// there is only one possible match for a string
	fn := func(s string, v interface{}) bool {
		out, pos = s, v
		return false
	}
	r.WalkPrefix(subString, fn)
	return out.(string), pos
}
\end{lstlisting}

% size, kB: 20, 50, 60, 70, 80, 90, 100, 120, 150
% time, ns: 220, 242, 243, 237, 243, 244, 243, 245, 256
% mem, KB:  212243, 1334534, 1918371, 2595544, 3370035, 4237366, 5210522, 7449706, 11529887

\begin{table}[h!]
    \centering
    \begin{tabular}{|c|c|c|}
        \hline
        Text size, Kb & Memory, Mb & Time, s\\
        \hline
        150 & 11529 & 256\\
        \hline
        120 & 7449 & 245\\
        \hline
        100 & 5210 & 243\\
        \hline
        90 & 4237 & 244\\
        \hline
        80 & 3370 & 243\\
        \hline
        70 & 2595 & 237\\
        \hline
        60 & 1918 & 243\\
        \hline
        50 & 1334 & 242\\
        \hline
        20 & 212 & 220\\
        \hline
    \end{tabular}
    \caption{Substring search in radix tree}
    \label{table:5}
\end{table}
