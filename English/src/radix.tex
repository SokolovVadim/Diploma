Реализация radix tree на языке Go взята из следующего источника: \cite{golang2016sa}
Во многом выбор обусловлен удобством встраивания бенчмарка в существующую систему тестирования на языке Go.
При этом с алгоритмической точки зрения эта реализация подтверждает
теоретическое описание сложности поиска подстроки за $O(1)$.
\newpage
\begin{lstlisting}[caption=Radix tree example]
func BenchmarkConstruct(b *testing.B, testStr string) {
	var substringArray []string = createSubstrings(testStr)
	b.ResetTimer()
	r := radix.New()
	for i := 0; i < b.n; i++ {
		fillRadixTree(b.N, r, substringArray)
	}
}
\end{lstlisting}

На приведенном листинге 2 в кратком виде показана функция построения radix tree.
В начале генерируется набор подстрок из текста, прочитанного из файла. В функции $fillRadixTree$
происходит добавление подстрок в дерево. Сравнительные характеристики построения radix tree
для Amazon Text Corpora \cite{amazon2013text} приведены в таблице \ref{table:4}.

% size, kB: 196, 148, 98, 79, 49, 10, 5, 2
% time, s:  276.419, 128.494, 57.319, 36.786, 14.632, 0.845, 0.414, 0.284
% mem, KB:  20248160, 11501051, 5192071, 3353864, 1324283, 52803, 13528, 2314

\begin{table}[h!]
    \centering
    \begin{tabular}{|c|c|c|}
        \hline
        Размер текста, KB & Память, MB & Время, s\\
        \hline
        196 & 20248 & 276.419\\
        \hline
        148 & 11501 & 128.494\\
        \hline
        98 & 5192 & 57.319\\
        \hline
        79 & 3353 & 36.786\\
        \hline
        49 & 1324 & 14.632\\
        \hline
        10 & 52 & 0.845\\
        \hline
        5 & 13 & 0.414\\
        \hline
        2 & 2 & 0.284\\
        \hline
    \end{tabular}
    \caption{Построение radix tree}
    \label{table:4}
\end{table}

Ниже в листинге 3 приведен упрощенный пример функции поиска подстроки в дереве.
Построенная структура показывает очень хороший результат поиска подстроки за константное время.
Его можно видеть в таблице \ref{table:5}.

\newpage
\begin{lstlisting}[caption=Radix tree lookup]
func getSubstring(r *radix.Tree,
                  subString string, testStr string) {
	var out interface{}
	var pos interface{}
	// there is only one possible match for a string
	fn := func(s string, v interface{}) bool {
		out, pos = s, v
		return false
	}
	r.WalkPrefix(subString, fn)
	return out.(string), pos
}
\end{lstlisting}

% size, kB: 20, 50, 60, 70, 80, 90, 100, 120, 150
% time, ns: 220, 242, 243, 237, 243, 244, 243, 245, 256
% mem, KB:  212243, 1334534, 1918371, 2595544, 3370035, 4237366, 5210522, 7449706, 11529887

\begin{table}[h!]
    \centering
    \begin{tabular}{|c|c|c|}
        \hline
        Размер текста, KB & Память, MB & Время, s\\
        \hline
        150 & 11529 & 256\\
        \hline
        120 & 7449 & 245\\
        \hline
        100 & 5210 & 243\\
        \hline
        90 & 4237 & 244\\
        \hline
        80 & 3370 & 243\\
        \hline
        70 & 2595 & 237\\
        \hline
        60 & 1918 & 243\\
        \hline
        50 & 1334 & 242\\
        \hline
        20 & 212 & 220\\
        \hline
    \end{tabular}
    \caption{Поиск подстроки в radix tree}
    \label{table:5}
\end{table}
