% SA
% Radix
% CSA
% Comparison

Considered data structure, compressed suffix array (CSA) showed it's efficiency
in terms of memory used for index storage. In comparison to traditional
suffix array as it was demonstrated at the graphs of comparable characteristics,
with a large enough size of the initial text CSA consumes less memory than suffix array
while preserving the opportunity to perform a substring lookup.
As it was mentioned at the discussion of dependency between compression from the text lenght,
at small initial text size suffix array is more effective in terms of memory consumption that
is caused by additional costs for maintaining supplementary data structures used for succinct index construction.
At the same time an algorithm of index construction as well as a substring lookup in CSA
work slower than in the classic one. Nevertheless, there are several methods of the
speed increase in these algorithms allowsing CSA to achieve results that are not worse
than suffix array ones \cite{andersensimple}.

The results obtained by the comparison of radix tree with other studied data structures
confirm the efficiency of substring lookup in radix tree in constant time.
However, text indexation requires fulfillment of the tree by substrings of the initial text,
that takes significantly more memory than suffix array and CSA.
Similarly, filling speed for a tree at the construction does not allow one to consider
radix tree as an effective method for operations with substrings of the initial text.
