% SA
% Radix
% CSA
% Comparison

A considered data structure, the compressed suffix array (CSA) showed its efficiency
in terms of memory used for index storage. In comparison to traditional
suffix array as it was demonstrated in the graphs of comparable characteristics,
with a large enough size of the initial text, CSA consumes less memory than suffix array
while preserving the opportunity to perform a substring lookup.
As was mentioned in the discussion of dependency between compression and the text length,
at a small initial text size suffix array is more effective in terms of memory consumption that
is caused by additional costs for maintaining supplementary data structures used for succinct index construction.
At the same time, an algorithm of index construction, as well as a substring lookup in CSA,
work slower than in the classic one. Nevertheless, there are several methods of 
the speed increase in these algorithms allowing CSA to achieve results that are not worse
than suffix array ones \cite{andersensimple}.

The results obtained by the comparison of the radix tree with other studied data structures
confirm the efficiency of substring lookup in the radix tree in constant time.
However, text indexation requires fulfillment of the tree by substrings of the initial text,
that takes significantly more memory than suffix array and CSA.
Similarly, the filling speed for a tree construction does not allow one to consider
radix tree as an effective method for operations with substrings of the initial text.
